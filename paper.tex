\documentclass{article}

\usepackage{microtype}
\usepackage{graphicx}
\usepackage{subfigure}
\usepackage{booktabs}

\usepackage{hyperref}

\newcommand{\theHalgorithm}{\arabic{algorithm}}

% Use the following line for the initial blind version submitted for review:
%\usepackage{sysml2019}

% If accepted, instead use the following line for the camera-ready submission:
\usepackage[accepted]{sysml2019}

\sysmltitlerunning{Guild AI: what to put?}

\begin{document}

\twocolumn[
\sysmltitle{Guild AI: what to put?}

% List of affiliations: The first argument should be a (short)
% identifier you will use later to specify author affiliations
% Academic affiliations should list Department, University, City, Region, Country
% Industry affiliations should list Company, City, Region, Country

% You can specify symbols, otherwise they are numbered in order.
% Ideally, you should not use this facility. Affiliations will be numbered
% in order of appearance and this is the preferred way.
\sysmlsetsymbol{equal}{*}

\begin{sysmlauthorlist}
\sysmlauthor{Garrett Smith}{equal,gai}
\end{sysmlauthorlist}

\sysmlaffiliation{gai}{Guild AI, Chicago, Illinois, USA}

\sysmlcorrespondingauthor{Garrett Smith}{garrett@guild.ai}

\sysmlkeywords{Machine Learning, SysML, Guild AI}

\vskip 0.3in

\begin{abstract}
  TODO - Tight summary of Guild, its value proposition wrt
  accelerating research, and the proposed demo audience experience.
\end{abstract}
]

% this must go after the closing bracket ] following \twocolumn[ ...

% This command actually creates the footnote in the first column
% listing the affiliations and the copyright notice.
% The command takes one argument, which is text to display at the start of the footnote.
% The \sysmlEqualContribution command is standard text for equal contribution.
% Remove it (just {}) if you do not need this facility.

%\printAffiliationsAndNotice{}  % leave blank if no need to mention equal contribution
\printAffiliationsAndNotice{\sysmlEqualContribution} % otherwise use the standard text.

\section{Guild AI}

TODO: What is Guild AI? What is the benefit?

Benefits:

- Trivializes reproducibility - saves time, let's researcher focus on
novel work

- Baselines are easy to integrate into a project

\subsection{Users}

TODO: Who is Guild intended for?

\subsection{Related Work}

TODO: What other projects

- Model DB

- Pip

- ???

\subsection{Differences}

TODO: What makes Guild unique?

- Simplicity and ease of use (talk about YAML file)

- Immediate benefit to users (value of reproducibility extends to
original researcher as well as surveyors)

- Consistent with traditional methods of artifact reuse (packages,
file system based, etc.)

\section{Demonstration}

TODO: Describe the format, approach, etc.

\subsection{Audience Experience}

TODO: What will the audience experience? What will they see? How will
they feel inside?

\subsection{Equipment}

TODO: What sort of equipment is needed?

\bibliography{paper}
\bibliographystyle{sysml2019}

\end{document}
